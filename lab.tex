%!TEX root = ../hall.tex
% Тип документа
\documentclass[a4paper,12pt]{extarticle}

% Шрифты, кодировки, символьные таблицы, переносы
% \usepackage{cmap}
% \usepackage[T2A]{fontenc}
\usepackage[utf8]{inputenc}
\usepackage[russian]{babel}
% Это пакет -- хитрый пакет, он нужен но не нужен
\usepackage[mode=buildnew]{standalone}

\usepackage
	{
		% Дополнения Американского математического общества (AMS)
		amssymb,
		amsfonts,
		amsmath,
		amsthm,
		% Пакет для физических текстов
		physics,
		% misccorr,
		% 
		% Графики и рисунки
		wrapfig,
		graphicx,
		subcaption,
		float,
		tikz,
		tikz-3dplot,
		caption,
		csvsimple,
		color,
		booktabs,
		geometry,
		% 
		% Таблицы, списки
		makecell,
		multirow,
		indentfirst,
		%
		% Интегралы и прочие обозначения
		ulem,
		esint,
		esdiff,
		% 
		% Колонтитулы
		fancyhdr,
	}  
\usepackage{pgfplots,pgfplotstable,booktabs,colortbl}
\usepackage{xcolor}
\usepackage{hyperref}
\usepackage{pythontex}
 % Цвета для гиперссылок
\definecolor{linkcolor}{HTML}{000000} % цвет ссылок
\definecolor{urlcolor}{HTML}{799B03} % цвет гиперссылок
 
\hypersetup{pdfstartview=FitH,linkcolor=linkcolor,urlcolor=urlcolor, colorlinks=true}
\hypersetup{pageanchor=false}
% Увеличенный межстрочный интервал, французские пробелы
\linespread{1.3} 
\frenchspacing 

 
% \usetikzlibrary
% 	{
% 		decorations.pathreplacing,
% 		decorations.pathmorphing,
% 		patterns,
% 		calc,
% 		scopes,
% 		arrows,
% 		fadings,
% 		through,
% 		shapes.misc,
% 		arrows.meta,
% 		3d,
% 		quotes,
% 		angles,
% 		babel
% 	}
% Среднее <#1>
\newcommand{\mean}[1]{\langle#1\rangle}

\begin{pycode}
##
def frexp10(decimal):
	parts = ('%e' % decimal).split('e')
	return float(parts[0]), int(parts[1])
##
\end{pycode}



% Функция для тех, кто использует pythontex. Представляет любое вещественное число в стандартном виде.
\newcommand{\frexp}[1]{
		\pyc{#10=frexp10(#1)} 
			\py{ round(#10[0],2)} 
				\cdot 10^{\py{#10[1]}} }

% const прямым шрифтом
\newcommand\ct[1]{\text{\rmfamily\upshape #1}}
\newcommand*{\const}{\ct{const}}
\usepackage{array}
\usepackage{pstool}

\geometry		
	{
		left			=	2cm,
		right 			=	2cm,
		top 			=	2.5cm,
		bottom 			=	2.5cm,
		bindingoffset	=	0cm
	}

%%%%%%%%%%%%%%%%%%%%%%%%%%%%%%%%%%%%%%%%%%%%%%%%%%%%%%%%%%%%%%%%%%%%%%%%%%%%%%%
	%применим колонтитул к стилю страницы
\pagestyle{fancy} 
	%очистим "шапку" страницы
% \fancyhead{} 
	%слева сверху на четных и справа на нечетных
\fancyhead[R]{}%\labauthors 
	%справа сверху на четных и слева на нечетных
% \fancyhead[L]{Отчёт по лабораторной работе №\labnumber}
\fancyhead[L]{\labtheme} 
	%очистим "подвал" страницы
% \fancyfoot{} 
	% номер страницы в нижнем колинтуле в центре
\fancyfoot[C]{\thepage} 

%%%%%%%%%%%%%%%%%%%%%%%%%%%%%%%%%%%%%%%%%%%%%%%%%%%%%%%%%%%%%%%%%%%%%%%%%%%%%%%

\renewcommand{\contentsname}{Оглавление}
\usepackage{tocloft}
\usepackage{secdot}
\sectiondot{subsection}


\begin{pycode}
##
from main import *
##
\end{pycode}
\usepackage{gensymb}
\usepackage{textcomp}
\usepackage{mathrsfs}
\begin{document}
\def\labauthors{Войтович Д.А., Понур К.А.}
\def\labgroup{440}
\def\department{Кафедра электродинамики}
\def\labnumber{1}
\def\labtheme{Электромагнитное экранирование}

\renewcommand{\phi}{\varphi}

\def\E{\mathscr{E}_H}
\def\Rdim{\,\frac{\text{м}^3}{\text{А} \cdot \text{с}}}
\begin{titlepage}

\begin{center}

{\small\textsc{Нижегородский государственный университет имени Н.\,И. Лобачевского}}
\vskip 1pt \hrule \vskip 3pt
{\small\textsc{Радиофизический факультет. Кафедра Электродинамики.}}

\vfill

{\Large Отчет по лабораторной работе №\labnumber\vskip 12pt\bfseries \labtheme}
	
\end{center}

\vfill
	
\begin{flushright}
	{Выполнили студенты \labgroup\ группы\\ \labauthors}%\vskip 12pt Принял:\\ Менсов С.\,Н.}
\end{flushright}
	
\vfill
	
\begin{center}
	Нижний Новгород, \the\year
\end{center}

\end{titlepage}


\tableofcontents
\newpage
\section*{Цели работы}

Настоящая работа преследует две основные цели.
\begin{itemize}
        \item Экспериментальное наблюдение явления экранирования переменного магнитного поля металлическими оболочками и выяснение роли основных физических факторов, определяющих степень проникновения поля через экран; к числу таких факторов относятся: свойства материала экрана (проводимость и магнитная проницаемость), толщина его стенок, частота поля.
        \item Теоретический расчет экранирующих свойств металлических оболочек на простой модели и сопоставление экспериментальных и теоретических данных.
\end{itemize}
\section{Элементы теории}%
\subsection{Основные понятия}%
\label{sub:2.1}
Под электромагнитным экранированием понимается изоляция некоторой
области пространства от проникновения электромагнитах полей, существующих в соседних областях. 
В статических или переменных квазистационарных полях (которым соответствуют длины волн, много больше характерных размеров используемых приборов и устройств) такая изоляция осуществляется обычно с помощью замкнутых металлических оболочек -- экранов. Явление экранирования поля проводящими оболочками имеет большое практическое значение. В частности, оно широко используется в электро- и радиотехнике для уменьшения паразитных связей между различными элементами приборов. В некоторых случаях,, напротив, может возникнуть необходимость принимать специальные меры для борьбы с этим явлением.

Общей физической причиной ослабления поля внутри экрана является то обстоятельство, что наведенные в нем внешним полем токи (или заряды) создают во внутренней области поле, противоположное внешнему. В результате суммарное поле в этой области, складывающиеся из полей внешних и наведенных источников, уменьшается. 

\subsection{Расчет экранирующего действия металлических оболочек}%
\label{sub:2.2}

В качестве экранов в работе используются оболочки цилиндрической формы.
Строгий расчет их экранирующего действия представлял бы собой весьма сложную задачу, требующую использования численных методов. Однако, для получения качественных оценок ослабления поля в экранированной области и установления общего характера его зависимости от параметров можно ограничиться изучением более простых моделей, допускающих точное решение задачи в известных аналитических функциях. Моделями такого рода являются, например, плоский, цилиндрический и сферический слои. Поскольку высота и диаметр внутренней полости используемых в работе экранирующих цилиндров одинаковы и весьма малы по сравнению с длиной волны в свободном пространстве $\lambda_{0}$, наиболее адекватной моделью, по-видимому, следует считать сферический слой, который имеет тот же объем внутренней полости и внешний радиус $a\ll \lambda_{0}$. Последнее условие означает, что вне металла, т.е. как во внешней, так и в экранируемой областях) поле можно рассматривать как квазистатическое. Подробное решение задачи об экранирующих свойствах сферического слоя по отношению к переменному магнитному полю дано в Приложении, помещенном в конце данного описания.
Ниже приведены основные результаты этого решения.

Если замкнутая однородная сферическая оболочка помещена в квазистатическое внешнее поле с комплексным вектором напряженности $\vec H_{0} e^{i \omega t}$, которое в её отсутствие является однородным, то поле в ограничиваемой ею области $\vec H_{1} e^{i \omega t }$ также однородно. Эффективности экранирования удобно характеризовать величиной отношения комплексных амплитуд этих полей:
\begin{equation}
    \label{eq:1}
    \eta_m = \frac{H_{0}}{H_{1}}.
\end{equation}
Величина $\abs{ \eta_M}$ показывает, в какое число раз ослабляется поле в экранированной области, и может быть названа коэффициентов ослабления. Она, естественно, сильно зависит от соотношения между толщиной экрана $d$ и толщиной скин-слоя $\delta = \frac{c}{\sqrt{2\pi\sigma\mu\omega}}$ ($c$ -- скорость света в вакууме, $\sigma$ -- проводимость, $\mu$-- магнитная проницаемость экрана). В двух предельных случаях ($\delta\ll d$ и $\delta\gg d$ ) выражение для $\eta_m$ (в общем случае довольно громоздкое) сильно упрощается и при выполнении дополнительного условия $d\ll a$ принимает следующий вид
\paragraph{$\delta\ll d$ (сильный скин-эффект)}%
\begin{equation}
    \label{eq:2}
    \eta_m= \frac{1}{6} \qty[ (1-i) \frac{\mu \delta}{a} + 3 + (1+i) \frac{a}{\mu \delta}]
    \exp[(1+i) \frac{d}{\delta}].
\end{equation}
При $\mu=1$ 
\begin{equation}
    \label{eq:3}
    \eta_m = \frac{1}{6} (1+i) \frac{a}{\delta} \exp[(1+i) \frac{d}{\delta}].
\end{equation}
\paragraph{$\delta\gg d$ (скин-эффект отсутствует}%
\begin{equation}
    \label{eq:4}
    \eta_m = 1 + \frac{2}{3 } \frac{d}{a} \frac{(\mu-1)^2}{\mu} + i \frac{2}{3} \frac{ad}{\mu\delta^2}.
\end{equation}
При $\mu=1$ 
\begin{equation}
    \label{eq:5}
    \eta_m = 1 + i \frac{2ad}{3\delta^2}.
\end{equation}

Для приближенных оценок величины $\eta_m$ (с точностью $\sim 10 \%$ выражения  
\eqref{eq:2}- \eqref{eq:5} можно использовать и в промежуточном случае ( $\delta\simeq d$ ), разграничивая области применимости формул \eqref{eq:2}, \eqref{eq:3}, с одной стороны,
и \eqref{eq:4}, \eqref{eq:5}, с другой стороны, точкой $\delta=d$.

Заметим, что приведенные результаты расчета позволяют описать также экранирующее действие металлической оболочки по отношению к переменному электрическому полю. В частности, при $\delta \gg a$ выражение для комплексного коэффициента ослабления электрического поля $\eta_e$ легко получается на основании принципа перестановочной двойственности из выражения \eqref{eq:4} путем замены в нем магнитной проницаемости $\mu$ на диэлектрическую проницаемость проводника $\epsilon = \frac{4 \pi \sigma}{i \omega}$. В диапазоне радиочастот величина $\abs{\epsilon}$ для хороших проводников и определяемая ею величина $\abs{\eta_e}$ принимают чрезвычайно высокие значения, недоступные для измерений в условиях настоящей работы даже при весьма малой толщине экранов. Например, при $\frac{d}{a}\simeq 10^{-3},\sigma \simeq 10^{17} \text{ с}^{-1}, \omega\simeq 10^{4} \text{ с}^{-1}$, пренебрегая в \eqref{eq:4} малыми членами и заменяя $\mu$ на $\epsilon$,
получаем:
\begin{equation}
    \label{eq:6}
    \eta_e = \frac{2\epsilon d}{3a} = \frac{-i {8} \pi \sigma d}{3\omega a} \simeq  - i \cdot 10^{11}.
\end{equation}
В полном соответствии с законами электростатики при $ \omega \to 0$ величина $\eta_e \to \infty$, т.е. электрическое поле внутрь экрана не проникает.

\section{Практическая часть}%
\label{sec:prakticheskaia_chast_}



\begin{thebibliography}{}
	\bibitem{lit0} Сарафанов Ф.Г. Блог <<\href{http://fedorsarafanov.github.io}{Physics \& other}>>. Н.Новгород: РФФ ННГУ, 2019.
	\bibitem{lit1} Ландау Л.Д., Лифшиц Е.М. Электродинамика сплошных сред -- М.: Физматлиц, 2005. -- \S\S56-61, задачи №1 к \S59 и №5 к \S86.
    \bibitem{lit2} Нейман Л.Р., Димирчан К.С., Юринов В.М. Руководство к лаборатории электромагнитного поля. -- М.: Высшая школа, 1966. -- Работа №11.
    \bibitem{lit3} Каден Г. Электромагнитные экраны в высокочастотной технике и электросвязи. --М.:Госэнергоиздат, 1957. --327 с.
\end{thebibliography}


\end{document}
